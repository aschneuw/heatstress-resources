\chapter{Abstract}
Climate change is increasingly contributing to higher average temperatures which lead to frequent and severe heat waves across the globe. These shifts can intensify heat stress in dairy cows and might adversely affect milk production and component yields. However, the degree of heat tolerance among dairy cow breeds has not been extensively studied in commercial production settings within Switzerland. Existing research has largely focused on a limited number of breeds in grassland-based systems outside of Switzerland. Using a comprehensive dataset comprising over 130 million records from the three major dairy cow breeding associations in Switzerland, we investigated the effect of weather on milk yield and energy-corrected milk yield across six dairy cow breeds, providing an unparalleled analysis in terms of scale and scope.

\vspace*{\baselineskip}
We employed Generalized Additive Mixed Models (GAMMs) to adequately model the non-linear Temperature Humidity Index (THI) and other variables such as days in milk (DIM). This required us to develop a modified implementation of \textit{gamm4 (R)} and \textit{MixedModels.jl (Julia)} adapted to the sparse structure of our dataset. With these modifications, we enabled GAMMs to accommodate a larger number of random effect factor levels than previous implementations in \textit{R} while enhancing computational efficiency.

\vspace*{\baselineskip}
Applying this tailored method to our model, we determined the marginal non-linear effects of THI on the daily milk yield and the daily energy-corrected milk (ECM) yield for both primiparous and multiparous cows. We retrieved breed-specific THI response curves and peak performance points. Our results indicated that dairy performance decreased at three-day mean THI values as low as 51 for certain breeds, which was notably lower than reported in comparable studies. Furthermore, we observed greater variability in peak THI values for milk yield across breeds compared to ECM yield, suggesting that high-fat and high-protein-producing breeds such as Jersey experience earlier declines in milk component yields. Across all scenarios, primiparous cows consistently showed lower heat tolerance compared to multiparous cows. Furthermore, when the data was divided into periods before and after 2010, lower peak THI values were observed in the latter period, possibly indicating a decline in heat tolerance due to breeding practices.

\vspace*{\baselineskip}
Despite substantial advancements in computational capabilities enabling the acquisition of our results, we advise conducting additional statistical validation employing techniques such as bootstrapping, cross-validation, or alternative subsampling strategies. Furthermore, to statistically consolidate the findings, consideration may be given to a multi-breed model that integrates all dairy cow breeds into a single unified model. Some of these steps would require additional algorithmic optimizations. Nonetheless, the rich dataset available presents opportunities for further research, including the examination of additional milk performance parameters such as lactose content or somatic cell count. Also, incorporating the animal breeding history into the model could provide deeper insights into breeding dynamics and effects on heat responses. The consideration of pre-calving weather exposure also warrants attention. Moreover, the low peak-performance THI values in our results highlight the necessity of future studies on heat stress mitigation strategies, for instance, through dietary interventions.
